The protocol defines four basic commands that manipulate spreadsheets as 
a unit: \hyperref[sec:message:create]{create}, \hyperref[sec:message:delete]{delete}, 
\hyperref[sec:message:rename]{rename}, and 
\hyperref[sec:message:list]{list\_spreadsheet}. 
Every spreadsheet is identified by an \emph{id} and associated with a \emph{name} that 
need not be unique. The \emph{name} is primarily for display purposes, while the 
spreadsheet \emph{id} is used to identify it in all operations pertaining to a 
specific spreadsheet.

\subsubsection{Creating a new spreadsheet}
The create command is used to create a new spreadsheet. Create expects a \emph{name} 
field to be provided to serve as the name of the new spreadsheet. Any errors 
resulting from this command will be purely due to an error occurring on the 
server, for example, if the server does not have the capacity for a new 
spreadsheet.

\subsubsection{Create Command}

In addition to the \emph{command} field which will have a value of "create", this command expects the following fields:
\begin{table}[H]
    \begin{center}
        \begin{tabular}{|c|c|c|}\hline
            Field & Type & Description \\\hline
            name & string & The name of the newly created spreadsheet. \\\hline
        \end{tabular}
    \end{center}
\end{table}

The following is an example of a \emph{create} command:
\lstinputlisting[caption={create command message},label={lst:command:create}]{Sections/commands/examples/create.json}

\subsubsection{Create Results}
The result of this command will contain the \hyperref[sec:message:result]{standard result} fields, in addition to the following fields to be included for \underline{ok} results:
\begin{table}[H]
    \begin{center}
        \begin{tabular}{|c|c|c|}\hline
            Field & Type & Description \\\hline
            id & number & The id of the newly created spreadsheet. \\\hline
        \end{tabular}
    \end{center}
\end{table}

The following are examples of results of the \emph{create} command:

\lstinputlisting[caption={create result (ok) message},label={lst:result:ok:create}]{Sections/commands/examples/create_result_ok.json}

\lstinputlisting[caption={create result (error) message},label={lst:result:err:create}]{Sections/commands/examples/create_result_err.json}


\subsubsection{Deleting a spreadsheet}
% TODO
Lorem ipsum dolor sit amet, consectetur adipiscing elit, sed do eiusmod tempor incididunt ut labore et dolore magna aliqua. Ut enim ad minim veniam, quis nostrud exercitation ullamco laboris nisi ut aliquip ex ea commodo consequat. Duis aute irure dolor in reprehenderit in voluptate velit esse cillum dolore eu fugiat nulla pariatur. Excepteur sint occaecat cupidatat non proident, sunt in culpa qui officia deserunt mollit anim id est laborum.

\subsubsection{Delete Command}
In addition to the \emph{command} field which will have a value of "delete", this command expects the following fields:

\begin{table}[H]
    \begin{center}
        \begin{tabular}{|c|c|c|}\hline
            Field & Type & Description \\\hline
            id & number & The \emph{id} of the spreadsheet to delete. \\\hline
        \end{tabular}
    \end{center}
\end{table}

The following is an example of a \emph{delete} command:

\lstinputlisting[caption={delete command message},label={lst:command:delete}]{Sections/commands/examples/delete.json}

\subsubsection{Delete Results}
The result of this command will contain the \hyperref[sec:message:result]{standard result} fields.
The following are examples of results of the \emph{delete} command:

\lstinputlisting[caption={delete result (ok) message},label={lst:result:ok:delete}]{Sections/commands/examples/delete_result_ok.json}

\lstinputlisting[caption={delete result (error) message},label={lst:result:err:delete}]{Sections/commands/examples/delete_result_err.json}



\subsubsection{Renaming a spreadsheet}
To rename a spreadsheet the client will use the \hyperref[sec:message:rename]{rename}
command, which accepts the \emph{id} of the spreadsheet to rename and the \emph{name} 
to rename the spreadsheet to. This command will trigger an update for all clients who 
currently have the spreadsheet \hyperref[sec:message:open]{open}, notifying them that 
the name of the spreadsheet has changed. See \hyperref[sec:message:rename]{rename} for 
a description of the command.

\subsubsection{Listing all spreadsheets}
The list\_spreadsheets command is used to retrieve the ids of all the 
spreadsheets stored on the server. This command will only fail if an error 
occurs on the server.

\subsubsection{List Spreadsheets Command}
The \emph{command} field which will have a value of "list\_spreadsheets", and will the \emph{Command} will not need to contain any other fields.

The following is an example of a \emph{list\_spreadsheets} command:
\lstinputlisting[caption={list command message},label={lst:command:list}]{Sections/commands/examples/list.json}

\subsubsection{List Spreadsheets Results}
The result of this command will contain the \hyperref[sec:message:result]{standard result} fields, in addition to the following fields to be included for \underline{ok} results:
\begin{table}[H]
    \begin{center}
        \begin{tabular}{|c|c|c|}\hline
            Field & Type & Description \\\hline
            spreadsheets & number[] & The \emph{id}s of all the spreadsheets. \\\hline
        \end{tabular}
    \end{center}
\end{table}

The following are examples of results of the \emph{list\_spreadsheets} command:

\lstinputlisting[caption={list result (ok) message},label={lst:result:ok:list}]{Sections/commands/examples/list_result_ok.json}

\lstinputlisting[caption={list result (error) message},label={lst:result:err:list}]{Sections/commands/examples/list_result_err.json}
