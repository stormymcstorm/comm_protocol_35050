% The client may want to revert the spreadsheet to a previous state, that is, 
% reapplying an earlier edit to the spreadsheet. This can be accomplished 
% through the \hyperref[sec:message:get_history]{get\_history} and \hyperref[sec:message:push]{push} 
% commands. The client must simply get the history of the desired spreadsheet, 
% select the edit to revert to, and then push that edit to the server as if it 
% were a new edit. Since, \hyperref[sec:message:get_history]{get\_history} 
% returns edits that have been undone, it is possible to revert an undo.

The client may want to revert a specific cell (or set of cells) to a previous 
state rather than undo entire edits. The protocol does not provide a specific 
command for this, rather it is the responsibility of the client to determine 
the most recent value of a cell (by using a cached spreadsheet history or by 
a new \hyperref[lst:command:get_history]{get\_history}) and then to \hyperref[lst:command:push]{push} 
an appropriate edit to the server. 
Implementing reverts in this way allows us to use a similar approach for 
reverting while offline