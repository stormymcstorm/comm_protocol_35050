% TODO
Lorem ipsum dolor sit amet, consectetur adipiscing elit, sed do eiusmod tempor incididunt ut labore et dolore magna aliqua. Ut enim ad minim veniam, quis nostrud exercitation ullamco laboris nisi ut aliquip ex ea commodo consequat. Duis aute irure dolor in reprehenderit in voluptate velit esse cillum dolore eu fugiat nulla pariatur. Excepteur sint occaecat cupidatat non proident, sunt in culpa qui officia deserunt mollit anim id est laborum.

\subsubsection{Push Command}
In addition to the \emph{command} field which will have a value of "push", this command expects the following fields:
\begin{table}[H]
    \begin{center}
        \begin{tabular}{|c|c|c|}\hline
            Field & Type & Description \\\hline
            id & number & The \emph{id} of the spreadsheet to perform the push on. \\\hline
            edits & \hyperref[sec:message:edits]{Edit}[] & The edits to apply to the spreadsheet. \\\hline
        \end{tabular}
    \end{center}
\end{table}

The following is an example of a \emph{push} command:
\lstinputlisting[caption={push command message},label={lst:command:push}]{Sections/commands/examples/push.json}

\subsubsection{Push Results}
The result of this command will contain the \hyperref[sec:message:result]{standard result} fields.
The following are examples of results of the \emph{push} command:
\lstinputlisting[caption={push result (ok) message},label={lst:result:ok:push}]{Sections/commands/examples/push_result_ok.json}

\lstinputlisting[caption={push result (error) message},label={lst:result:err:push}]{Sections/commands/examples/push_result_err.json}
