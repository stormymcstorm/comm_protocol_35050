The \emph{push} command is used to push edits to the server so that they can 
be propagated to all other clients. The pushed edits will be merged into the 
spreadsheet according to the provided \emph{from\_id}, which specifies the 
point in spreadsheet’s history these edits where made. The server will handle 
the merging of histories, to allow for synchronizing after going offline. 
Pushed edits must be valid in that they do not cause a circular dependency or 
formula error. Push will trigger an update to be sent to all clients who have 
the spreadsheet open informing them that an edit has been made. This command 
can fail if there is no spreadsheet that matches the provided \emph{id}, the 
pushed edits are not valid or if an error occurs on the server.

\subsubsection{Push Command}
In addition to the \emph{command} field which will have a value of ``push", this command expects the following fields:
\begin{table}[H]
    \begin{center}
        \begin{tabular}{|c|c|c|}\hline
            Field & Type & Description \\\hline
            id & number & The \emph{id} of the spreadsheet to perform the push on. \\\hline
            from\_id & number & The \emph{id} of the edit preceding the provided edits. \\\hline
            edits & \hyperref[sec:message:edits]{Edit}[] & The edits to apply to the spreadsheet. \\\hline
        \end{tabular}
    \end{center}
\end{table}
The \emph{from\_id} specifies the id of the most recent edit received from the server, this allows clients to push changes
based on an earlier version of the spreadsheet. This edit id will most likely come from spreadsheet \hyperref[lst:update:edits]{updates},
a call to \hyperref[sec:message:get_spreadsheet]{get\_spreadsheet}

The following is an example of a \emph{push} command:
\lstinputlisting[caption={push command message},label={lst:command:push}]{Sections/commands/examples/push.json}

\subsubsection{Push Results}
The result of this command will contain the \hyperref[sec:message:result]{standard result} fields.
The following are examples of results of the \emph{push} command:
\lstinputlisting[caption={push result (ok) message},label={lst:result:ok:push}]{Sections/commands/examples/push_result_ok.json}

\lstinputlisting[caption={push result (error) message},label={lst:result:err:push}]{Sections/commands/examples/push_result_err.json}
