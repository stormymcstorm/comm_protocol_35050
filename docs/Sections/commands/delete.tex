The \emph{delete} command is used to delete an existing spreadsheet. Delete will 
erase all the spreadsheet’s history and remove it from the spreadsheet list. 
This command can fail if there is no spreadsheet that matches the provided \emph{id}, 
or if an error occurs on the server. Deleting a spreadsheet will trigger an \hyperref[lst:update:delete]{update} 
to be sent to all clients who have the spreadsheet open informing them that 
the spreadsheet has been deleted.

\subsubsection{Delete Command}
In addition to the \emph{command} field which will have a value of ``delete", this command expects the following fields:

\begin{table}[H]
    \begin{center}
        \begin{tabular}{|c|c|c|}\hline
            Field & Type & Description \\\hline
            id & number & The \emph{id} of the spreadsheet to delete. \\\hline
        \end{tabular}
    \end{center}
\end{table}

The following is an example of a \emph{delete} command:

\lstinputlisting[caption={delete command message},label={lst:command:delete}]{Sections/commands/examples/delete.json}

\subsubsection{Delete Results}
The result of this command will contain the \hyperref[sec:message:result]{standard result} fields.
The following are examples of results of the \emph{delete} command:

\lstinputlisting[caption={delete result (ok) message},label={lst:result:ok:delete}]{Sections/commands/examples/delete_result_ok.json}

\lstinputlisting[caption={delete result (error) message},label={lst:result:err:delete}]{Sections/commands/examples/delete_result_err.json}

