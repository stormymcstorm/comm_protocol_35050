The \emph{create} command is used to create a new spreadsheet. Create expects a \emph{name} 
field to be provided to serve as the name of the new spreadsheet. Any errors 
resulting from this command will be purely due to an error occurring on the 
server, for example, if the server does not have the capacity for a new 
spreadsheet.

\subsubsection{Create Command}

In addition to the \emph{command} field which will have a value of ``create", this command expects the following fields:
\begin{table}[H]
    \begin{center}
        \begin{tabular}{|c|c|c|}\hline
            Field & Type & Description \\\hline
            name & string & The name of the newly created spreadsheet. \\\hline
        \end{tabular}
    \end{center}
\end{table}

The following is an example of a \emph{create} command:
\lstinputlisting[caption={create command message},label={lst:command:create}]{Sections/commands/examples/create.json}

\subsubsection{Create Results}
The result of this command will contain the \hyperref[sec:message:result]{standard result} fields, in addition to the following fields to be included for \underline{ok} results:
\begin{table}[H]
    \begin{center}
        \begin{tabular}{|c|c|c|}\hline
            Field & Type & Description \\\hline
            id & number & The id of the newly created spreadsheet. \\\hline
        \end{tabular}
    \end{center}
\end{table}

The following are examples of results of the \emph{create} command:

\lstinputlisting[caption={create result (ok) message},label={lst:result:ok:create}]{Sections/commands/examples/create_result_ok.json}

\lstinputlisting[caption={create result (error) message},label={lst:result:err:create}]{Sections/commands/examples/create_result_err.json}
