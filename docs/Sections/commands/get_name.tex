The get\_name command is used to retrieve the name of a spreadsheet given the id of the spreadsheet. This command can fail if there is no spreadsheet that matches the provided id, or if an error occurs on the server.

\subsubsection{Get Name Command}

In addition to the \emph{command} field which will have a value of "get\_name", this command expects the following fields:

\begin{table}[H]
    \begin{center}
        \begin{tabular}{|c|c|c|}\hline
            Field & Type & Description \\\hline
            id & number & The \emph{id} of the spreadsheet to whose name to get. \\\hline
        \end{tabular}
    \end{center}
\end{table}

The following is an example of a \emph{get\_name} command:

\lstinputlisting[caption={get\_name command message},label={lst:command:get_name}]{Sections/commands/examples/get_name.json}

\subsubsection{Get Name Results}
The result of this command will contain the \hyperref[sec:message:result]{standard result} fields, in addition to the following fields to be included for \underline{ok} results:
\begin{table}[H]
    \begin{center}
        \begin{tabular}{|c|c|c|}\hline
            Field & Type & Description \\\hline
            name & string & The \emph{name} of requested spreadsheet. \\\hline
        \end{tabular}
    \end{center}
\end{table}

The following are examples of results of the \emph{get\_name} command:

\lstinputlisting[caption={get\_name result (ok) message},label={lst:result:ok:get_name}]{Sections/commands/examples/get_name_result_ok.json}

\lstinputlisting[caption={get\_name result (error) message},label={lst:result:err:get_name}]{Sections/commands/examples/get_name_result_err.json}
