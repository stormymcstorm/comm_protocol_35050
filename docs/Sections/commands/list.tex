The \emph{list\_spreadsheets} command is used to retrieve the ids of all the 
spreadsheets stored on the server. This command will only fail if an error 
occurs on the server.

\subsubsection{List Spreadsheets Command}
The \emph{command} field which will have a value of "list\_spreadsheets", and will the \emph{Command} will not need to contain any other fields.

The following is an example of a \emph{list\_spreadsheets} command:
\lstinputlisting[caption={list command message},label={lst:command:list}]{Sections/commands/examples/list.json}

\subsubsection{List Spreadsheets Results}
The result of this command will contain the \hyperref[sec:message:result]{standard result} fields, in addition to the following fields to be included for \underline{ok} results:
\begin{table}[H]
    \begin{center}
        \begin{tabular}{|c|c|c|}\hline
            Field & Type & Description \\\hline
            spreadsheets & string[] & The \emph{name}s of the spreadsheets, where the index is their id. \\\hline
        \end{tabular}
    \end{center}
\end{table}

The following are examples of results of the \emph{list\_spreadsheets} command:

\lstinputlisting[caption={list result (ok) message},label={lst:result:ok:list}]{Sections/commands/examples/list_result_ok.json}

\lstinputlisting[caption={list result (error) message},label={lst:result:err:list}]{Sections/commands/examples/list_result_err.json}
