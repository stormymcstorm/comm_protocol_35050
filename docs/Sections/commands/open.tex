The \emph{open} command is used to subscribe to updates for a specific spreadsheet. 
This command will cause the server to add the current \href{https://en.wikipedia.org/wiki/Transmission_Control_Protocol}{TCP} 
connection with the client to a list of subscribers that will receive updates 
associated with a specific sheet. If a client fails to receive any update, it 
will be removed from the subscribers list and will need to reopen the 
spreadsheet to begin receiving updates again. The server will respond to the 
open command with the current contents of the spreadsheet in the same way it 
does for the \hyperref[sec:message:get_spreadsheet]{get\_spreadsheet} command. 
This command can fail if there is no spreadsheet that matches the provided \emph{id}, 
or if an error occurs on the server.

\subsubsection{Open Command}
In addition to the \emph{command} field which will have a value of ``open", this command expects the following fields:
\begin{table}[H]
    \begin{center}
        \begin{tabular}{|c|c|c|}\hline
            Field & Type & Description \\\hline
            id & number & The \emph{id} of spreadsheet to open. \\\hline
        \end{tabular}
    \end{center}
\end{table}

The following is an example of a \emph{open} command:
\lstinputlisting[caption={open command message},label={lst:command:open}]{Sections/commands/examples/open.json}

\subsubsection{Open Results}
The result of this command will contain the \hyperref[sec:message:result]{standard result} fields, in addition to the following fields to be included for \underline{ok} results:
\begin{table}[H]
    \begin{center}
        \begin{tabular}{|c|c|c|}\hline
            Field & Type & Description \\\hline
            current\_id & number & The \emph{id} of most recent edit to the spreadsheet. \\\hline
            spreadsheet & \hyperref[sec:message:spreadsheet]{Spreadsheet} & The contents of the opened spreadsheet. \\\hline
        \end{tabular}
    \end{center}
\end{table}

The following are examples of results of the \emph{open} command:
\lstinputlisting[caption={open result (ok) message},label={lst:result:ok:open}]{Sections/commands/examples/open_result_ok.json}

\lstinputlisting[caption={open result (error) message},label={lst:result:err:open}]{Sections/commands/examples/open_result_err.json}

\subsubsection{Updates}
\label{sec:message:updates}
There are three variants of the updates that clients will receive for the spreadsheets they have open: \emph{delete}, \emph{rename} and \emph{edits}.
Each update message will contain an \emph{update} field which will identify the type of update and an \emph{id} field which will identify the spreadsheet
the update is for. The \emph{delete} update will contain only the \emph{update} and \emph{id} fields which indicate that the spreadsheet has been deleted.
The \emph{rename} update will contain the standard fields, as well as a \emph{name} field which will contain the new name of the spreadsheet. The \emph{edits}
update will contain a \emph{edits} field which will contain an array of \hyperref[sec:message:edits]{Edit} objects in order from oldest to newest edits.
The update will contain all edits that have been made since the last update. The update will also contain a \emph{start\_id} field which will contain the edit id,
of the first edit, each of the following edit ids can be computed by incrementing the edit id.
The following are examples of the three variants of the update messages:

\lstinputlisting[caption={delete update message},label={lst:update:delete}]{Sections/commands/examples/update_delete.json}
\lstinputlisting[caption={rename update message},label={lst:update:rename}]{Sections/commands/examples/update_rename.json}
\lstinputlisting[caption={edits update message},label={lst:update:edits}]{Sections/commands/examples/update_edits.json}