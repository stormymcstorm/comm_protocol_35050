The \emph{rename} command is used to rename an existing spreadsheet. Rename will 
change the \emph{name} that is associated with the spreadsheet. This command 
can fail if there is no spreadsheet that matches the provided \emph{id}, or 
if an error occurs on the server. Renaming a spreadsheet will trigger an 
\hyperref[lst:update:rename]{update} to be sent to all clients who have the 
spreadsheet open, informing them that the name of the spreadsheet has changed.

\subsubsection{Rename Command}
In addition to the \emph{command} field which will have a value of ``rename", this command expects the following fields:

\begin{table}[H]
    \begin{center}
        \begin{tabular}{|c|c|c|}\hline
            Field & Type & Description \\\hline
            id & number & The \emph{id} of the spreadsheet to rename. \\\hline
            name & string & The new \emph{name} of the spreadsheet. \\\hline
        \end{tabular}
    \end{center}
\end{table}

The following is an example of a \emph{rename} command:
\lstinputlisting[caption={rename command message},label={lst:command:rename}]{Sections/commands/examples/rename.json}

\subsubsection{Rename Results}
The result of this command will contain the \hyperref[sec:message:result]{standard result} fields.
The following are examples of results of the \emph{rename} command:

\lstinputlisting[caption={rename result (ok) message},label={lst:result:ok:rename}]{Sections/commands/examples/rename_result_ok.json}

\lstinputlisting[caption={rename result (error) message},label={lst:result:err:rename}]{Sections/commands/examples/rename_result_err.json}

