The \emph{undo} command is used to undo the most recent edit. This command 
will rollback the history of the spreadsheet, but will not erase the edit 
that was rolled-back so that it can still be accessed by \hyperref[sec:message:get_history]{get\_history}. 
Undo will trigger an \hyperref[lst:update:edits]{update} to be sent to all 
clients who have the spreadsheet open informing them that an edit has been 
made. While an undo is not actually an edit, for the purpose of updating 
the clients, undoes will be treated as if the reapply the previous edit. 
This command can fail if there is no spreadsheet that matches the provided \emph{id}, 
or if an error occurs on the server.

\subsubsection{Undo Command}
In addition to the \emph{command} field which will have a value of ``undo", this command expects the following fields:
\begin{table}[H]
    \begin{center}
        \begin{tabular}{|c|c|c|}\hline
            Field & Type & Description \\\hline
            id & number & The \emph{id} of the spreadsheet to perform the undo on. \\\hline
        \end{tabular}
    \end{center}
\end{table}

The following is an example of a \emph{undo} command:
\lstinputlisting[caption={undo command message},label={lst:command:undo}]{Sections/commands/examples/undo.json}

\subsubsection{Undo Results}
The result of this command will contain the \hyperref[sec:message:result]{standard result} fields.
The following are examples of results of the \emph{undo} command:
\lstinputlisting[caption={undo result (ok) message},label={lst:result:ok:undo}]{Sections/commands/examples/undo_result_ok.json}

\lstinputlisting[caption={undo result (error) message},label={lst:result:err:undo}]{Sections/commands/examples/undo_result_err.json}
