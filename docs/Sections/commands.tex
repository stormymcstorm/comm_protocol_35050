% Table describing all of the commands
\begin{table}[H]
    \begin{center}
        \begin{tabular}{|c|c|}\hline
        Command & Description \\\hline
        \hyperref[sec:message:create]{create} & Creates a new spreadsheet \\\hline
        \hyperref[sec:message:delete]{delete} & Deletes a spreadsheet \\\hline
        \hyperref[sec:message:rename]{rename} & Renames a spreadsheet \\\hline
        \hyperref[sec:message:list]{list\_spreadsheets} & Lists all spreadsheets \\\hline
        \hyperref[sec:message:open]{open} & Subscribes the client to updates \\\hline
        \hyperref[sec:message:close]{close} & Unsubscribes the client to updates \\\hline
        \hyperref[sec:message:get_history]{get\_history} & Gets the edit history of a spreadsheet \\\hline
        \hyperref[sec:message:get_spreadsheet]{get\_spreadsheet} & Gets the contents of a spreadsheet \\\hline
        \hyperref[sec:message:push]{push} & Pushes edits to be applied to the spreadsheet \\\hline
        \hyperref[sec:message:undo]{undo} & Undoes the most recent edit of a spreadsheet \\\hline
        \end{tabular}
    \end{center}
\end{table}

\lstset{language=json,frame=single,numbers=none,captionpos=b}

\subsection{Common Data Types}

\subsubsection{Command}
\label{sec:message:command}

All \emph{Command}s contain the \emph{command} field which is used to identify the command being executed.
All \emph{Command}s will contain the following fields:

\begin{table}[H]
    \begin{center}
        \begin{tabular}{|c|c|c|}\hline
            Field & Type & Description \\\hline
            command & string & The command to execute \\\hline
            $\ldots$ & any & Command parameters \\\hline
        \end{tabular}
    \end{center}
\end{table}

\subsubsection{Result}
\label{sec:message:result}

There server will respond to each \emph{Command} with a \emph{Result}. \emph{Result}s have two 
variants: \emph{ok} and \emph{error}. All \emph{Result}s will contain the following fields:

\begin{table}[H]
    \begin{center}
        \begin{tabular}{|c|c|c|}\hline
            Field & Type & Description \\\hline
            result & string & Identifies the command this result is for \\\hline
            ok & boolean & Whether or not the command was successful \\\hline
        \end{tabular}
    \end{center}
\end{table}

If the \emph{Result} is ok ($ok = true$) then the \emph{Result} may contain addition fields
that contain the results of the command. Otherwise the \emph{Result} will contain the following fields:

\begin{table}[H]
    \begin{center}
        \begin{tabular}{|c|c|c|}\hline
            Field & Type & Description \\\hline
            error & string & A description of the error that occurred. \\\hline
        \end{tabular}
    \end{center}
\end{table}

\subsubsection{Edits and Spreadsheet}
\label{sec:message:edits}
\label{sec:message:spreadsheet}

Various commands will accept as parameters or return as a result the contents of a spreadsheet,
both \emph{Edit}s and \emph{Spreadsheet}s will contain spreadsheet contents. They differ in that
a \emph{Spreadsheet} object must contain the contents of a whole spreadsheet while a \emph{Edit}
object may contain a subset of the contents of a spreadsheet. Both \emph{Edit}s and \emph{Spreadsheet}s
will contain the following fields:

\begin{table}[H]
    \begin{center}
        \begin{tabular}{|c|c|c|}\hline
            Field & Type & Description \\\hline
            \$CellName & string & The contents of \$CellName. \\\hline
        \end{tabular}
    \end{center}
\end{table}

These data types are simply a mapping between cell names and their contents, valid cell names are a
capital letter from "A" to "Z" followed by a number from 1 to 99.

\subsection{Create}
\label{sec:message:create}
To create a new spreadsheet the client will use the \emph{create} command, 
which accepts the name of the spreadsheet to create, and responds with the 
\emph{id} of the newly created spreadsheet. The newly created spreadsheet 
will persist on the server until it is deleted. See \hyperref[sec:message:create]{create} 
for a description of the \emph{create} command.

% TODO: Insert diagram here

\subsection{Delete}
\label{sec:message:delete}
% TODO
Lorem ipsum dolor sit amet, consectetur adipiscing elit, sed do eiusmod tempor incididunt ut labore et dolore magna aliqua. Ut enim ad minim veniam, quis nostrud exercitation ullamco laboris nisi ut aliquip ex ea commodo consequat. Duis aute irure dolor in reprehenderit in voluptate velit esse cillum dolore eu fugiat nulla pariatur. Excepteur sint occaecat cupidatat non proident, sunt in culpa qui officia deserunt mollit anim id est laborum.

\subsubsection{Delete Command}
In addition to the \emph{command} field which will have a value of "delete", this command expects the following fields:

\begin{table}[H]
    \begin{center}
        \begin{tabular}{|c|c|c|}\hline
            Field & Type & Description \\\hline
            id & number & The \emph{id} of the spreadsheet to delete. \\\hline
        \end{tabular}
    \end{center}
\end{table}

The following is an example of a \emph{delete} command:

\lstinputlisting[caption={delete command message},label={lst:command:delete}]{Sections/commands/examples/delete.json}

\subsubsection{Delete Results}
The result of this command will contain the \hyperref[sec:message:result]{standard result} fields.
The following are examples of results of the \emph{delete} command:

\lstinputlisting[caption={delete result (ok) message},label={lst:result:ok:delete}]{Sections/commands/examples/delete_result_ok.json}

\lstinputlisting[caption={delete result (error) message},label={lst:result:err:delete}]{Sections/commands/examples/delete_result_err.json}



\subsection{Rename}
\label{sec:message:rename}
Lorem ipsum dolor sit amet, consectetur adipiscing elit, sed do eiusmod tempor incididunt ut labore et dolore magna aliqua. Ut enim ad minim veniam, quis nostrud exercitation ullamco laboris nisi ut aliquip ex ea commodo consequat. Duis aute irure dolor in reprehenderit in voluptate velit esse cillum dolore eu fugiat nulla pariatur. Excepteur sint occaecat cupidatat non proident, sunt in culpa qui officia deserunt mollit anim id est laborum.

\subsection{List Spreadsheets}
\label{sec:message:list}
The \emph{list\_spreadsheets} command is used to retrieve the ids of all the 
spreadsheets stored on the server. This command will only fail if an error 
occurs on the server.

\subsubsection{List Spreadsheets Command}
The \emph{command} field which will have a value of "list\_spreadsheets", and will the \emph{Command} will not need to contain any other fields.

The following is an example of a \emph{list\_spreadsheets} command:
\lstinputlisting[caption={list command message},label={lst:command:list}]{Sections/commands/examples/list.json}

\subsubsection{List Spreadsheets Results}
The result of this command will contain the \hyperref[sec:message:result]{standard result} fields, in addition to the following fields to be included for \underline{ok} results:
\begin{table}[H]
    \begin{center}
        \begin{tabular}{|c|c|c|}\hline
            Field & Type & Description \\\hline
            spreadsheets & number[] & The \emph{id}s of all the spreadsheets. \\\hline
        \end{tabular}
    \end{center}
\end{table}

The following are examples of results of the \emph{list\_spreadsheets} command:

\lstinputlisting[caption={list result (ok) message},label={lst:result:ok:list}]{Sections/commands/examples/list_result_ok.json}

\lstinputlisting[caption={list result (error) message},label={lst:result:err:list}]{Sections/commands/examples/list_result_err.json}


\subsection{Open}
\label{sec:message:open}
% TODO
Lorem ipsum dolor sit amet, consectetur adipiscing elit, sed do eiusmod tempor incididunt ut labore et dolore magna aliqua. Ut enim ad minim veniam, quis nostrud exercitation ullamco laboris nisi ut aliquip ex ea commodo consequat. Duis aute irure dolor in reprehenderit in voluptate velit esse cillum dolore eu fugiat nulla pariatur. Excepteur sint occaecat cupidatat non proident, sunt in culpa qui officia deserunt mollit anim id est laborum.

\subsubsection{Open Command}
In addition to the \emph{command} field which will have a value of "open", this command expects the following fields:
\begin{table}[H]
    \begin{center}
        \begin{tabular}{|c|c|c|}\hline
            Field & Type & Description \\\hline
            id & number & The \emph{id} of spreadsheet to open. \\\hline
        \end{tabular}
    \end{center}
\end{table}

The following is an example of a \emph{open} command:
\lstinputlisting[caption={open command message},label={lst:command:open}]{Sections/commands/examples/open.json}

\subsubsection{Open Results}
The result of this command will contain the \hyperref[sec:message:result]{standard result} fields, in addition to the following fields to be included for \underline{ok} results:
\begin{table}[H]
    \begin{center}
        \begin{tabular}{|c|c|c|}\hline
            Field & Type & Description \\\hline
            spreadsheet & \hyperref[sec:message:spreadsheet]{Spreadsheet} & The contents of the opened spreadsheet. \\\hline
        \end{tabular}
    \end{center}
\end{table}

The following are examples of results of the \emph{open} command:
\lstinputlisting[caption={open result (ok) message},label={lst:result:ok:open}]{Sections/commands/examples/open_result_ok.json}

\lstinputlisting[caption={open result (error) message},label={lst:result:err:open}]{Sections/commands/examples/open_result_err.json}


\subsection{Close}
\label{sec:message:close}
In the event that a client no longer wishes to receive updates for a 
spreadsheet, for example, if the user closes the client or the spreadsheet, 
the client can unsubscribe from spreadsheet updates by using the \hyperref[sec:message:close]{close} 
command, which accepts the \emph{id} of the spreadsheet the client wishes to 
close. This will prevent the client from receiving further updates about the 
spreadsheet, and the client will need to reopen the spreadsheet to begin 
receiving updates again.


\subsection{Get History}
\label{sec:message:get_history}
The \emph{get\_history} command is used to retrieve the history of a spreadsheet. 
This command will return the complete history of the spreadsheet separated 
into edits ordered from oldest to newest. This returned history may contain 
edits that have been undone, so any edit that comes after the current edit 
has been undone. This command can fail if there is no spreadsheet that matches 
the provided \emph{id}, or if an error occurs on the server.

\subsubsection{Get History Command}
In addition to the \emph{command} field which will have a value of "get\_history", this command expects the following fields:
\begin{table}[H]
    \begin{center}
        \begin{tabular}{|c|c|c|}\hline
            Field & Type & Description \\\hline
            id & number & The \emph{id} of the spreadsheet whose history to get \\\hline
        \end{tabular}
    \end{center}
\end{table}

The following is an example of a \emph{get\_history} command:
\lstinputlisting[caption={get\_history command message},label={lst:command:get_history}]{Sections/commands/examples/get_history.json}

\subsubsection{Get History Results}
The result of this command will contain the \hyperref[sec:message:result]{standard result} fields, in addition to the following fields to be included for \underline{ok} results:
\begin{table}[H]
    \begin{center}
        \begin{tabular}{|c|c|c|}\hline
            Field & Type & Description \\\hline
            edits & \hyperref[sec:message:edits]{Edit}[] & The edit history in order from oldest to newest. \\\hline
            current & number & The zeroth based index of the current state of the spreadsheet. \\\hline
        \end{tabular}
    \end{center}
\end{table}

The following are examples of results of the \emph{get\_history} command:
\lstinputlisting[caption={get\_history result (ok) message},label={lst:result:ok:get_history}]{Sections/commands/examples/get_history_result_ok.json}

\lstinputlisting[caption={get\_history result (error) message},label={lst:result:err:get_history}]{Sections/commands/examples/get_history_result_err.json}


\subsection{Get Spreadsheet}
\label{sec:message:get_spreadsheet}
% TODO
Lorem ipsum dolor sit amet, consectetur adipiscing elit, sed do eiusmod tempor incididunt ut labore et dolore magna aliqua. Ut enim ad minim veniam, quis nostrud exercitation ullamco laboris nisi ut aliquip ex ea commodo consequat. Duis aute irure dolor in reprehenderit in voluptate velit esse cillum dolore eu fugiat nulla pariatur. Excepteur sint occaecat cupidatat non proident, sunt in culpa qui officia deserunt mollit anim id est laborum.

\lstinputlisting[caption={get\_spreadsheet command message},label={lst:command:get_spreadsheet}]{Sections/commands/examples/get_spreadsheet.json}

\lstinputlisting[caption={get\_spreadsheet result (ok) message},label={lst:result:ok:get_spreadsheet}]{Sections/commands/examples/get_spreadsheet_result_ok.json}

\lstinputlisting[caption={get\_spreadsheet result (error) message},label={lst:result:err:get_spreadsheet}]{Sections/commands/examples/get_spreadsheet_result_err.json}


\subsection{Undo}
\label{sec:message:undo}
% TODO
Lorem ipsum dolor sit amet, consectetur adipiscing elit, sed do eiusmod tempor incididunt ut labore et dolore magna aliqua. Ut enim ad minim veniam, quis nostrud exercitation ullamco laboris nisi ut aliquip ex ea commodo consequat. Duis aute irure dolor in reprehenderit in voluptate velit esse cillum dolore eu fugiat nulla pariatur. Excepteur sint occaecat cupidatat non proident, sunt in culpa qui officia deserunt mollit anim id est laborum.

\subsubsection{Undo Command}
In addition to the \emph{command} field which will have a value of "undo", this command expects the following fields:
\begin{table}[H]
    \begin{center}
        \begin{tabular}{|c|c|c|}\hline
            Field & Type & Description \\\hline
            id & number & The \emph{id} of the spreadsheet to perform the undo on. \\\hline
        \end{tabular}
    \end{center}
\end{table}

The following is an example of a \emph{undo} command:
\lstinputlisting[caption={undo command message},label={lst:command:undo}]{Sections/commands/examples/undo.json}

\subsubsection{Undo Results}
The result of this command will contain the \hyperref[sec:message:result]{standard result} fields.
The following are examples of results of the \emph{undo} command:
\lstinputlisting[caption={undo result (ok) message},label={lst:result:ok:undo}]{Sections/commands/examples/undo_result_ok.json}

\lstinputlisting[caption={undo result (error) message},label={lst:result:err:undo}]{Sections/commands/examples/undo_result_err.json}


\subsection{Push}
\label{sec:message:push}
% TODO
Lorem ipsum dolor sit amet, consectetur adipiscing elit, sed do eiusmod tempor incididunt ut labore et dolore magna aliqua. Ut enim ad minim veniam, quis nostrud exercitation ullamco laboris nisi ut aliquip ex ea commodo consequat. Duis aute irure dolor in reprehenderit in voluptate velit esse cillum dolore eu fugiat nulla pariatur. Excepteur sint occaecat cupidatat non proident, sunt in culpa qui officia deserunt mollit anim id est laborum.

\subsubsection{Push Command}
In addition to the \emph{command} field which will have a value of "push", this command expects the following fields:
\begin{table}[H]
    \begin{center}
        \begin{tabular}{|c|c|c|}\hline
            Field & Type & Description \\\hline
            id & number & The \emph{id} of the spreadsheet to perform the push on. \\\hline
            edits & \hyperref[sec:message:edits]{Edit}[] & The edits to apply to the spreadsheet. \\\hline
        \end{tabular}
    \end{center}
\end{table}

The following is an example of a \emph{push} command:
\lstinputlisting[caption={push command message},label={lst:command:push}]{Sections/commands/examples/push.json}

\subsubsection{Push Results}
The result of this command will contain the \hyperref[sec:message:result]{standard result} fields.
The following are examples of results of the \emph{push} command:
\lstinputlisting[caption={push result (ok) message},label={lst:result:ok:push}]{Sections/commands/examples/push_result_ok.json}

\lstinputlisting[caption={push result (error) message},label={lst:result:err:push}]{Sections/commands/examples/push_result_err.json}

