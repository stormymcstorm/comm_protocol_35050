% Table describing all of the commands
\begin{table}[H]
    \begin{center}
        \begin{tabular}{|c|c|}\hline
        Command & Description \\\hline
        \hyperref[sec:message:create]{create} & Creates a new spreadsheet \\\hline
        \hyperref[sec:message:delete]{delete} & Deletes a spreadsheet \\\hline
        \hyperref[sec:message:rename]{rename} & Renames a spreadsheet \\\hline
        \hyperref[sec:message:get_name]{get\_name} & Gets the name of a spreadsheet \\\hline
        \hyperref[sec:message:list]{list\_spreadsheets} & Lists all spreadsheets \\\hline
        \hyperref[sec:message:open]{open} & Subscribes the client to updates \\\hline
        \hyperref[sec:message:close]{close} & Unsubscribes the client to updates \\\hline
        \hyperref[sec:message:get_history]{get\_history} & Gets the edit history of a spreadsheet \\\hline
        \hyperref[sec:message:get_spreadsheet]{get\_spreadsheet} & Gets the contents of a spreadsheet \\\hline
        \hyperref[sec:message:push]{push} & Pushes edits to be applied to the spreadsheet \\\hline
        \hyperref[sec:message:undo]{undo} & Undoes the most recent edit of a spreadsheet \\\hline
        \end{tabular}
    \end{center}
\end{table}

\lstset{language=json,frame=single,numbers=none,captionpos=b}

\subsection{Common Data Types}

\subsubsection{Command}
\label{sec:message:command}

All \emph{Command}s contain the \emph{command} field which is used to identify the command being executed.
All \emph{Command}s will contain the following fields:

\begin{table}[H]
    \begin{center}
        \begin{tabular}{|c|c|c|}\hline
            Field & Type & Description \\\hline
            command & string & The command to execute \\\hline
            $\ldots$ & any & Command parameters \\\hline
        \end{tabular}
    \end{center}
\end{table}

\subsubsection{Result}
\label{sec:message:result}

There server will respond to each \emph{Command} with a \emph{Result}. \emph{Result}s have two 
variants: \emph{ok} and \emph{error}. All \emph{Result}s will contain the following fields:

\begin{table}[H]
    \begin{center}
        \begin{tabular}{|c|c|c|}\hline
            Field & Type & Description \\\hline
            result & string & Identifies the command this result is for \\\hline
            ok & boolean & Whether or not the command was successful \\\hline
        \end{tabular}
    \end{center}
\end{table}

If the \emph{Result} is ok ($ok = true$) then the \emph{Result} may contain addition fields
that contain the results of the command. Otherwise the \emph{Result} will contain the following fields:

\begin{table}[H]
    \begin{center}
        \begin{tabular}{|c|c|c|}\hline
            Field & Type & Description \\\hline
            error & string & A description of the error that occurred. \\\hline
        \end{tabular}
    \end{center}
\end{table}

\subsubsection{Edits and Spreadsheet}
\label{sec:message:edits}
\label{sec:message:spreadsheet}

Various commands will accept as parameters or return as a result the contents of a spreadsheet,
both \emph{Edit}s and \emph{Spreadsheet}s will contain spreadsheet contents. They differ in that
a \emph{Spreadsheet} object must contain the contents of a whole spreadsheet while a \emph{Edit}
object may contain a subset of the contents of a spreadsheet. Both \emph{Edit}s and \emph{Spreadsheet}s
will contain the following fields:

\begin{table}[H]
    \begin{center}
        \begin{tabular}{|c|c|c|}\hline
            Field & Type & Description \\\hline
            \$CellName & string & The contents of \$CellName. \\\hline
        \end{tabular}
    \end{center}
\end{table}

These data types are simply a mapping between cell names and their contents, valid cell names are a
capital letter from "A" to "Z" followed by a number from 1 to 99.

\subsection{Create}
\label{sec:message:create}
The create command is used to create a new spreadsheet. Create expects a \emph{name} 
field to be provided to serve as the name of the new spreadsheet. Any errors 
resulting from this command will be purely due to an error occurring on the 
server, for example, if the server does not have the capacity for a new 
spreadsheet.

\subsubsection{Create Command}

In addition to the \emph{command} field which will have a value of "create", this command expects the following fields:
\begin{table}[H]
    \begin{center}
        \begin{tabular}{|c|c|c|}\hline
            Field & Type & Description \\\hline
            name & string & The name of the newly created spreadsheet. \\\hline
        \end{tabular}
    \end{center}
\end{table}

The following is an example of a \emph{create} command:
\lstinputlisting[caption={create command message},label={lst:command:create}]{Sections/commands/examples/create.json}

\subsubsection{Create Results}
The result of this command will contain the \hyperref[sec:message:result]{standard result} fields, in addition to the following fields to be included for \underline{ok} results:
\begin{table}[H]
    \begin{center}
        \begin{tabular}{|c|c|c|}\hline
            Field & Type & Description \\\hline
            id & number & The id of the newly created spreadsheet. \\\hline
        \end{tabular}
    \end{center}
\end{table}

The following are examples of results of the \emph{create} command:

\lstinputlisting[caption={create result (ok) message},label={lst:result:ok:create}]{Sections/commands/examples/create_result_ok.json}

\lstinputlisting[caption={create result (error) message},label={lst:result:err:create}]{Sections/commands/examples/create_result_err.json}


\subsection{Delete}
\label{sec:message:delete}
% TODO
Lorem ipsum dolor sit amet, consectetur adipiscing elit, sed do eiusmod tempor incididunt ut labore et dolore magna aliqua. Ut enim ad minim veniam, quis nostrud exercitation ullamco laboris nisi ut aliquip ex ea commodo consequat. Duis aute irure dolor in reprehenderit in voluptate velit esse cillum dolore eu fugiat nulla pariatur. Excepteur sint occaecat cupidatat non proident, sunt in culpa qui officia deserunt mollit anim id est laborum.

\subsubsection{Delete Command}
In addition to the \emph{command} field which will have a value of "delete", this command expects the following fields:

\begin{table}[H]
    \begin{center}
        \begin{tabular}{|c|c|c|}\hline
            Field & Type & Description \\\hline
            id & number & The \emph{id} of the spreadsheet to delete. \\\hline
        \end{tabular}
    \end{center}
\end{table}

The following is an example of a \emph{delete} command:

\lstinputlisting[caption={delete command message},label={lst:command:delete}]{Sections/commands/examples/delete.json}

\subsubsection{Delete Results}
The result of this command will contain the \hyperref[sec:message:result]{standard result} fields.
The following are examples of results of the \emph{delete} command:

\lstinputlisting[caption={delete result (ok) message},label={lst:result:ok:delete}]{Sections/commands/examples/delete_result_ok.json}

\lstinputlisting[caption={delete result (error) message},label={lst:result:err:delete}]{Sections/commands/examples/delete_result_err.json}



\subsection{Rename}
\label{sec:message:rename}
To rename a spreadsheet the client will use the \hyperref[sec:message:rename]{rename}
command, which accepts the \emph{id} of the spreadsheet to rename and the \emph{name} 
to rename the spreadsheet to. This command will trigger an update for all clients who 
currently have the spreadsheet \hyperref[sec:message:open]{open}, notifying them that 
the name of the spreadsheet has changed. See \hyperref[sec:message:rename]{rename} for 
a description of the command.

\subsection{Get Name}
\label{sec:message:get_name}
The get\_name command is used to retrieve the name of a spreadsheet given the id of the spreadsheet. This command can fail if there is no spreadsheet that matches the provided id, or if an error occurs on the server.

\subsubsection{Get Name Command}

In addition to the \emph{command} field which will have a value of "get\_name", this command expects the following fields:

\begin{table}[H]
    \begin{center}
        \begin{tabular}{|c|c|c|}\hline
            Field & Type & Description \\\hline
            id & number & The \emph{id} of the spreadsheet to whose name to get. \\\hline
        \end{tabular}
    \end{center}
\end{table}

The following is an example of a \emph{get\_name} command:

\lstinputlisting[caption={get\_name command message},label={lst:command:get_name}]{Sections/commands/examples/get_name.json}

\subsubsection{Get Name Results}
The result of this command will contain the \hyperref[sec:message:result]{standard result} fields, in addition to the following fields to be included for \underline{ok} results:
\begin{table}[H]
    \begin{center}
        \begin{tabular}{|c|c|c|}\hline
            Field & Type & Description \\\hline
            name & string & The \emph{name} of requested spreadsheet. \\\hline
        \end{tabular}
    \end{center}
\end{table}

The following are examples of results of the \emph{get\_name} command:

\lstinputlisting[caption={get\_name result (ok) message},label={lst:result:ok:get_name}]{Sections/commands/examples/get_name_result_ok.json}

\lstinputlisting[caption={get\_name result (error) message},label={lst:result:err:get_name}]{Sections/commands/examples/get_name_result_err.json}


\subsection{List Spreadsheets}
\label{sec:message:list}
The list\_spreadsheets command is used to retrieve the ids of all the 
spreadsheets stored on the server. This command will only fail if an error 
occurs on the server.

\subsubsection{List Spreadsheets Command}
The \emph{command} field which will have a value of "list\_spreadsheets", and will the \emph{Command} will not need to contain any other fields.

The following is an example of a \emph{list\_spreadsheets} command:
\lstinputlisting[caption={list command message},label={lst:command:list}]{Sections/commands/examples/list.json}

\subsubsection{List Spreadsheets Results}
The result of this command will contain the \hyperref[sec:message:result]{standard result} fields, in addition to the following fields to be included for \underline{ok} results:
\begin{table}[H]
    \begin{center}
        \begin{tabular}{|c|c|c|}\hline
            Field & Type & Description \\\hline
            spreadsheets & number[] & The \emph{id}s of all the spreadsheets. \\\hline
        \end{tabular}
    \end{center}
\end{table}

The following are examples of results of the \emph{list\_spreadsheets} command:

\lstinputlisting[caption={list result (ok) message},label={lst:result:ok:list}]{Sections/commands/examples/list_result_ok.json}

\lstinputlisting[caption={list result (error) message},label={lst:result:err:list}]{Sections/commands/examples/list_result_err.json}


\subsection{Open}
\label{sec:message:open}
The \emph{open} command is used to subscribe to updates for a specific spreadsheet. 
This command will cause the server to add the current \href{https://en.wikipedia.org/wiki/Transmission_Control_Protocol}{TCP} 
connection with the client to a list of subscribers that will receive updates 
associated with a specific sheet. If a client fails to receive any update, it 
will be removed from the subscribers list and will need to reopen the 
spreadsheet to begin receiving updates again. The server will respond to the 
open command with the current contents of the spreadsheet in the same way it 
does for the \hyperref[sec:message:get_spreadsheet]{get\_spreadsheet} command. 
This command can fail if there is no spreadsheet that matches the provided \emph{id}, 
or if an error occurs on the server.

\subsubsection{Open Command}
In addition to the \emph{command} field which will have a value of ``open", this command expects the following fields:
\begin{table}[H]
    \begin{center}
        \begin{tabular}{|c|c|c|}\hline
            Field & Type & Description \\\hline
            id & number & The \emph{id} of spreadsheet to open. \\\hline
        \end{tabular}
    \end{center}
\end{table}

The following is an example of a \emph{open} command:
\lstinputlisting[caption={open command message},label={lst:command:open}]{Sections/commands/examples/open.json}

\subsubsection{Open Results}
The result of this command will contain the \hyperref[sec:message:result]{standard result} fields, in addition to the following fields to be included for \underline{ok} results:
\begin{table}[H]
    \begin{center}
        \begin{tabular}{|c|c|c|}\hline
            Field & Type & Description \\\hline
            spreadsheet & \hyperref[sec:message:spreadsheet]{Spreadsheet} & The contents of the opened spreadsheet. \\\hline
        \end{tabular}
    \end{center}
\end{table}

The following are examples of results of the \emph{open} command:
\lstinputlisting[caption={open result (ok) message},label={lst:result:ok:open}]{Sections/commands/examples/open_result_ok.json}

\lstinputlisting[caption={open result (error) message},label={lst:result:err:open}]{Sections/commands/examples/open_result_err.json}

\subsubsection{Updates}
\label{sec:message:updates}
There are three variants of the updates that clients will receive for the spreadsheets they have open: \emph{delete}, \emph{rename} and \emph{edits}.
Each update message will contain an \emph{update} field which will identify the type of update and an \emph{id} field which will identify the spreadsheet
the update is for. The \emph{delete} update will contain only the \emph{update} and \emph{id} fields which indicate that the spreadsheet has been deleted.
The \emph{rename} update will contain the standard fields, as well as a \emph{name} field which will contain the new name of the spreadsheet. The \emph{edits}
update will contain a \emph{edits} field which will contain an array of \hyperref[sec:message:edits]{Edit} objects in order from oldest to newest edits.
The update will contain all edits that have been made since the last update. The update will also contain a \emph{start\_id} field which will contain the edit id,
of the first edit, each of the following edit ids can be computed by incrementing the edit id.
The following are examples of the three variants of the update messages:

\lstinputlisting[caption={delete update message},label={lst:update:delete}]{Sections/commands/examples/update_delete.json}
\lstinputlisting[caption={rename update message},label={lst:update:rename}]{Sections/commands/examples/update_rename.json}
\lstinputlisting[caption={edits update message},label={lst:update:edits}]{Sections/commands/examples/update_edits.json}

\subsection{Close}
\label{sec:message:close}
% TODO
Lorem ipsum dolor sit amet, consectetur adipiscing elit, sed do eiusmod tempor incididunt ut labore et dolore magna aliqua. Ut enim ad minim veniam, quis nostrud exercitation ullamco laboris nisi ut aliquip ex ea commodo consequat. Duis aute irure dolor in reprehenderit in voluptate velit esse cillum dolore eu fugiat nulla pariatur. Excepteur sint occaecat cupidatat non proident, sunt in culpa qui officia deserunt mollit anim id est laborum.

\subsubsection{Close Command}
In addition to the \emph{command} field which will have a value of "close", this command expects the following fields:
\begin{table}[H]
    \begin{center}
        \begin{tabular}{|c|c|c|}\hline
            Field & Type & Description \\\hline
            id & number & The \emph{id} of spreadsheet to close. \\\hline
        \end{tabular}
    \end{center}
\end{table}

The following is an example of a \emph{close} command:
\lstinputlisting[caption={close command message},label={lst:command:close}]{Sections/commands/examples/close.json}

\subsubsection{Close Results}
The result of this command will contain the \hyperref[sec:message:result]{standard result} fields.
The following are examples of results of the \emph{close} command:
\lstinputlisting[caption={close result (ok) message},label={lst:result:ok:close}]{Sections/commands/examples/close_result_ok.json}

\lstinputlisting[caption={close result (error) message},label={lst:result:err:close}]{Sections/commands/examples/close_result_err.json}


\subsection{Get History}
\label{sec:message:get_history}
% TODO
Lorem ipsum dolor sit amet, consectetur adipiscing elit, sed do eiusmod tempor incididunt ut labore et dolore magna aliqua. Ut enim ad minim veniam, quis nostrud exercitation ullamco laboris nisi ut aliquip ex ea commodo consequat. Duis aute irure dolor in reprehenderit in voluptate velit esse cillum dolore eu fugiat nulla pariatur. Excepteur sint occaecat cupidatat non proident, sunt in culpa qui officia deserunt mollit anim id est laborum.

\subsubsection{Get History Command}
In addition to the \emph{command} field which will have a value of "get\_history", this command expects the following fields:
\begin{table}[H]
    \begin{center}
        \begin{tabular}{|c|c|c|}\hline
            Field & Type & Description \\\hline
            id & number & The \emph{id} of the spreadsheet whose history to get \\\hline
        \end{tabular}
    \end{center}
\end{table}

The following is an example of a \emph{get\_history} command:
\lstinputlisting[caption={get\_history command message},label={lst:command:get_history}]{Sections/commands/examples/get_history.json}

\subsubsection{Get History Results}
The result of this command will contain the \hyperref[sec:message:result]{standard result} fields, in addition to the following fields to be included for \underline{ok} results:
\begin{table}[H]
    \begin{center}
        \begin{tabular}{|c|c|c|}\hline
            Field & Type & Description \\\hline
            edits & \hyperref[sec:message:edits]{Edit}[] & The edit history in order from oldest to newest. \\\hline
            current & number & The zeroth based index of the current state of the spreadsheet. \\\hline
        \end{tabular}
    \end{center}
\end{table}

The following are examples of results of the \emph{get\_history} command:
\lstinputlisting[caption={get\_history result (ok) message},label={lst:result:ok:get_history}]{Sections/commands/examples/get_history_result_ok.json}

\lstinputlisting[caption={get\_history result (error) message},label={lst:result:err:get_history}]{Sections/commands/examples/get_history_result_err.json}


\subsection{Get Spreadsheet}
\label{sec:message:get_spreadsheet}
In the event that the client wishes to retrieve the current state of the 
spreadsheet regardless of whether or not they’ve opened the spreadsheet, 
they can use the \hyperref[sec:message:get_spreadsheet]{get\_spreadsheet} 
command, which accepts the \emph{id} of the spreadsheet to get. The server 
will then respond with the contents of the spreadsheet.


\subsection{Undo}
\label{sec:message:undo}
Undo operations are not considered edits, but simply a rollback of the history 
of the spreadsheet. This protocol provides an \hyperref[sec:message:undo]{undo} 
command which will put the spreadsheet in the state it was before the most 
recent edit was applied. \emph{undo} accepts the \emph{id} of the spreadsheet 
to perform the undo on. This command will trigger an update for all clients 
that currently have the spreadsheet open.

% TODO: Discuss how the update will look


\subsection{Push}
\label{sec:message:push}
% TODO
Lorem ipsum dolor sit amet, consectetur adipiscing elit, sed do eiusmod tempor incididunt ut labore et dolore magna aliqua. Ut enim ad minim veniam, quis nostrud exercitation ullamco laboris nisi ut aliquip ex ea commodo consequat. Duis aute irure dolor in reprehenderit in voluptate velit esse cillum dolore eu fugiat nulla pariatur. Excepteur sint occaecat cupidatat non proident, sunt in culpa qui officia deserunt mollit anim id est laborum.

\subsubsection{Push Command}
In addition to the \emph{command} field which will have a value of "push", this command expects the following fields:
\begin{table}[H]
    \begin{center}
        \begin{tabular}{|c|c|c|}\hline
            Field & Type & Description \\\hline
            id & number & The \emph{id} of the spreadsheet to perform the push on. \\\hline
            from\_id & number & The \emph{id} of the edit preceding the provided edits. \\\hline
            edits & \hyperref[sec:message:edits]{Edit}[] & The edits to apply to the spreadsheet. \\\hline
        \end{tabular}
    \end{center}
\end{table}
The \emph{from\_id} specifies the id of the most recent edit received from the server, this allows clients to push changes
based on an earlier version of the spreadsheet. This edit id will most likely come from spreadsheet \hyperref[lst:update:edits]{updates},
a call to \hyperref[sec:message:get_spreadsheet]{get\_spreadsheet}

The following is an example of a \emph{push} command:
\lstinputlisting[caption={push command message},label={lst:command:push}]{Sections/commands/examples/push.json}

\subsubsection{Push Results}
The result of this command will contain the \hyperref[sec:message:result]{standard result} fields.
The following are examples of results of the \emph{push} command:
\lstinputlisting[caption={push result (ok) message},label={lst:result:ok:push}]{Sections/commands/examples/push_result_ok.json}

\lstinputlisting[caption={push result (error) message},label={lst:result:err:push}]{Sections/commands/examples/push_result_err.json}

